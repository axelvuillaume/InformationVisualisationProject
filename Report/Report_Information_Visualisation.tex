\documentclass{article}

\usepackage[ddmmyyyy]{datetime} 
\usepackage{caption}
\usepackage[backend=bibtex8]{biblatex}
\usepackage{hyperref}

\begin{document}
	\title{Report Information Visualisation}
	\author{Oruç Kaplan; Ward Van den Bossche; Axel Daniel Vuillaume \& Laurens Devloo}
	\maketitle
	\tableofcontents
	\newpage
	
	\section{Data}
	
	\subsection{Dataset}
	
	This project will use two different datasets
	
	\begin{itemize}
		\item \href{https://www.kaggle.com/datasets/fronkongames/steam-games-dataset/code}{Steam Game Data}
		\item Steam API for user information
	\end{itemize}
	
	The steam game dataset contains data about games received from the Steam API and Steam Spy. The contained columns are name of the game, the release date, the amount of DLC for the game, a short description of the game, is the game is on Linux, Windows or on Mac, a metacritic score, the amount of achievements the game has, user score the amount of negative reviews and positive reviews, an email address for help, an average playtime and the amount of owners of the game.\\
	\\
	The second dataset is user information gotten by the Steam API. This dataset will contain the data about a given Steam user. This information includes the games he plays and the user-profiles of his friends.
	
	\subsection{Target User}
	
	The target users are games, both the casual one as the more experienced one. The site will be able to recommend gamers games that might be interesting for them. It does this in two ways the first one is by showing the user general information about the games available. The second way is by giving insights in the current most popular games/genres or categories of the users and showing other games in the same vein. The user will also be able to see what is friends are playing to reap some inspirations in this way.
	
	\newpage
	
	\section{Preprocessing}
	
	\section{Visualisation}
	
	\section{Evaluation}
	
	
	\newpage
\end{document}